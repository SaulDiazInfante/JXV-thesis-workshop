\documentclass{beamer}
\usepackage[spanish, activeacute]{babel}
\usepackage{xcolor}
\usepackage{color}
\usepackage{colortbl}
\usepackage{amsmath}
\usepackage{amssymb}
\usepackage{graphicx}
\usepackage{latexsym}
\usepackage{ucs}
\usepackage[utf8]{inputenc}
\usepackage{siunitx}
\usepackage{times}
\usepackage{tikz}
\usepackage{verbatim}
\usepackage{pgf,pgfarrows,pgfnodes,pgfautomata,pgfheaps,pgfshade}
\usepackage{listings}
\usepackage{esint}
\usepackage{lipsum}
\usepackage{mathptmx}
\usepackage{helvet}
\usepackage{tikz}%
\usepackage{csvsimple}
\usepackage{pgfplots}
\usepackage{proba}
\usepackage[absolute, overlay]{textpos}
\usepackage{bibunits}
\usepackage{tcolorbox}
\usepackage{mathtools}
%\usepackage[
%    texcoord,
%    grid,
%    gridunit=mm,
%    gridcolor=red!60,
%    subgridcolor=gray!60]{eso-pic}
\usepackage{epigraph}
\setlength{\epigraphwidth}{.8\textwidth}
\usepackage{DejaVuSansMono}
%
\definecolor{mainthemecolour}{rgb}{0.42,0.48,0.37}
\definecolor{mainthemecolourlight}{rgb}{0.63,0.72,0.57}
\definecolor{mainthemecolourstrong}{rgb}{0.40,0.68,0.18}
\definecolor{mid-gray}{gray}{0.7}
%
\definecolor{greenstrong}{rgb}{0.58,0.77,0.29}
\definecolor{redstrong}{rgb}{0.81,0.22,0.23}
\definecolor{fglisting}{gray}{0.3}
\definecolor{bglisting}{gray}{1}
\definecolor{fgshell}{gray}{1}
\definecolor{bgshell}{gray}{0.1}
\definecolor{bgshelllight}{gray}{0.8}
%
%
\makeatletter
\let\@@magyar@captionfix\relax
\makeatother
%
% Some in-code macros - a bit buggy, but useful
\newcommand{\hl}[1]{\textcolor{greenstrong}{#1}}
\newcommand{\hlErr}[1]{\textcolor{redstrong}{\texttt{#1}}}
\newcommand{\hlOk}[1]{\textcolor{green}{\texttt{#1}}}
\newcommand{\hlInv}[1]{\colorbox{bgshell}{\textcolor{fgshell}{\texttt{#1}}}}
\newcommand{\unhl}[1]{\textcolor{gray}{#1}}
\newcommand{\clda}[0]{$\textcolor{blue}{\lambda}$}
\newcommand{\carr}[0]{$\textcolor{purple}{\rightarrow}$}
\newcommand{\cbind}[0]{\textbf{\texttt{$>\!\!>\!\!=$}}}
\newcommand{\codedots}[0]{\textcolor{mid-gray}{...}}
%%%%%%%%%%%%%%%%%%%%%%%%%%%%%%%%%%%%%%%%%%%%%%%%
\usetheme{elegance}
%%%%%%%%%%%%%%%%%%%%%%%%%%%%%%%%%%%%%%%%%%%%%%%%%%%%%%%%%%%%%%%%%%%%%%%%%%%%%%%%
\tcbuselibrary{skins, breakable}
%
\newtcolorbox{greenbox}[1]{%
        colback = green!5!white,
        colframe = green!55!black,
        fonttitle = \bfseries,
        title = #1
    }
\newtcolorbox{bluebox}[1]{%
        colback = blue!5!white,
        colframe = blue!55!black,
        fonttitle = \bfseries,
        title = #1
    }
%
\newtcolorbox{graybox}[1]{%
        colback = gray!5!white,
        colframe = gray!55!black,
        fonttitle = \bfseries,
        title = #1
    }
%
\newtcolorbox{yellowbox}[1]{%
        colback = yellow!5!white,
        colframe = yellow!55!black,
        fonttitle = \bfseries,
        title = #1
    }
%
\hypersetup{
    pdfpagemode = , % afficher le pdf en plein écran
    pdfauthor   = {Saul Diaz Infante Velasco},%
    pdftitle    = {Charla Biomatesis},%
    pdfsubject  = {Bio-matematicas},%
    pdfkeywords = {Science,Impact},%
    pdfcreator  = {PDFLaTeX,emacs,AucTeX, LuaLaTex},%
    pdfproducer = {CONACYT-Universidad de Sonora}%
}
%
\usetikzlibrary{arrows,shapes}
\newcommand{\tikzmark}[1]{\tikz[remember picture] \node[coordinate] (#1) {#1};}
%
% Fragile frames
\newenvironment{xframe}[1][]
  {\begin{frame}[fragile,environment=xframe,#1]}
  {\end{frame}}

\title{%
    Extinci\'on, Persistencia y Comportamiento % 
    \\ Umbral en Modelos Compartimentales %
}%
\subtitle{Estoc\'asticmente Perturbados}
\author{Sa\'ul D\'iaz Infante Velasco}
\institute{
    \color{white}
    CONACYT-Universidad de Sonora \\
    sauldiazinfante@gmail.com
} %
\date{
    \footnotesize
    \color{mainthemecolour} 
    UNAM, 
    Juriquilla, Queretaro  
    \\
    \today. }
%%%%%%%%%%%%%%%%%%%%%%%%%%%%%%%%%%%%%%%%%%%%%%%%%%%%%%%%%%%%%%%%%%%%%%%%%%%%%%%%
\makeatletter
\let\@@magyar@captionfix\relax
\makeatother
\input{setup}
\defaultbibliography{main}
\defaultbibliographystyle{abbrv}
%%%%%%%%%%%%%%%%%%%%%%%%%%%%%%%%%%%%%%%%%%%%%%%%%%%%%%%%%%%%%%%%%%%%%%%%%%%%%%%%
\begin{document}
    \maketitle
    \section{Introducci\'on}
        \begin{frame}
    \frametitle{Para fijar ideas}
    \begin{textblock*}{70mm}(5mm, 20mm)
        \begin{equation*}
            \begin{aligned}
                \dot{S}(t) &=
                    \Lambda - \mu S(t)
                    - \textcolor<4->{orange}{\beta}
                        S(t) I(t)
                    - \delta S(t)
                    \\
                \dot{I}(t) &=
                    \textcolor<4->{orange}{\beta}
                        S(t) I(t)
                    -(\mu + \gamma + \varepsilon) I(t)
                    \\
                \only<1-2>{
                    \dot{R}(t) &=
                        \gamma I(t)
                        - \mu R(t)
                        + \delta S(t)
                }
            \end{aligned}
        \end{equation*}
    \end{textblock*}
%%
    \only<2-7>{
        \begin{textblock*}{50mm}(5mm, 60mm)
            \begin{tcolorbox}[%
                space to upper,
                skin=bicolor,
                colbacklower=black!75,
                collower=white,
                title={Umbral determinista},
                halign=center,
                valign=center,
                bottom=2mm,
                height=35mm
            ]
                \begin{align*}
                    \mathcal{R}_0 &=
                        \frac{\beta \Lambda}{%
                            (\mu + \gamma + \varepsilon )%
                            (\mu + \delta)%
                        }%
                    \\
                    \mathcal{R}_0  & < 1
                        \ \Rightarrow
                            \ FDE: \text{ (g.a.s)}
                    \\
                    \mathcal{R}_0  & > 1
                        \ \Rightarrow
                            \ EE: \text{\quad (g.a.s)}
                \end{align*}
            \end{tcolorbox}
        \end{textblock*}
    }
    \only<5->{
        \begin{textblock*}{42mm}(80mm, 25mm)
             \begin{tcolorbox}
                 $
                     \beta d t \rightsquigarrow
                     \beta d t + \sigma dB_t
                 $
             \end{tcolorbox}
         \end{textblock*}
    }
%     %
    \only<6->{
        \begin{textblock*}{50mm}(40mm, 40mm)
            \begin{align*}
                \dot{S}(t) &=
                    \Lambda - \mu S(t)
                    - \beta S(t) I(t)
                    - \delta S(t)
                    -
                    \hl{%
                      \sigma S(t) I(t)
                       dB_t}
                    \\
                \dot{I}(t) &=
                    \beta S(t) I(t)
                    - (\mu + \gamma + \varepsilon) I(t)
                    + \hl{\sigma S(t) I(t) dB_t}
            \end{align*}
        \end{textblock*}
    }
    \only<7>{
        \begin{textblock*}{50mm}(70mm, 60mm)
            \begin{tcolorbox}[%
                space to upper,
                skin=bicolor,
                colbacklower=black!75,
                collower=white,
                title={Umbral estoc\'astico},
                halign=center,
                valign=center,
                bottom=2mm,
                height=35mm
            ]
                \begin{align*}
                    \mathcal{R}_0 ^ S &= \text{ ?}
                    \\
                    \mathcal{R}_0 ^ S &< 1
                    \ \Rightarrow \text{ extinci\'on}
                    \\
                    \mathcal{R}_0 ^ S &> 1
                    \ \Rightarrow \text{ persistencia}
                \end{align*}
            \end{tcolorbox}
        \end{textblock*}
    }
    \only<8>{
        \begin{textblock*}{105mm}(20mm, 57mm)
            \begin{tcolorbox}[title=Ver:]
                \begin{bibunit}[apalike]
                    \nocite{Zhang2018a}
                    \putbib
                \end{bibunit}
            \end{tcolorbox}
        \end{textblock*}
    }
\end{frame}

        \begin{frame}
    \frametitle{?`Cuando considerar Modelos Estoc\'asticos?}
    \begin{textblock*}{45mm}(3mm, 25mm)
        \begin{greenbox}{Sean importantes}
            \begin{itemize}
                \item
                    Poblaciones peque\~nas
                \item
                    \textcolor<2>{orange}{
                        Variabilidad demogr\'afica
                    }
                \item
                    \textcolor<3>{orange}{
                        Variabilidad ambiental
                    }
            \end{itemize}
        \end{greenbox}
    \end{textblock*}
    \begin{textblock*}{60mm}(60mm, 25mm)
        \begin{bluebox}{
            \only<1>{
                Seg\'un:
            }
            \only<2->{
                Ejemplo
            }%
        }
        \only<1>{
            \begin{bibunit}[apalike]
                \nocite{Allen2017}
                \putbib
            \end{bibunit}
        }
        \only<2->{
            Transmisi\'on, recuperaci\'on,
            nacimientos, muertes.
        }
        \only<3->{
            \tcblower
            Condiciones territoriales,
            acu\'aticas: enfermedades
            vectoriales, zoonóticas
            transmitidas por alimentos.
        }
        \end{bluebox}
    \end{textblock*}
\end{frame}

        \input{./introduction/alternativas.tex}
        \begin{frame}
    \frametitle{Objetivo}
    \begin{textblock*}{80mm}(25mm, 40mm)
            \begin{yellowbox}{%
                Ilustrar las ideas de 
                $
                    \varphi dt 
                    \rightsquigarrow 
                    \varphi dt 
                    + 
                    \sigma dB_t
                $
            }
                \begin{list}{$\bullet$}{}
                    \item
                        Modelación
                    \item
                        Análisis y Simulación 
                    \item
                        Ideas al aire
                \end{list}
            \end{yellowbox}
    \end{textblock*}
\end{frame}
        %
        \begin{frame}{Esquema de Charla}
            \setbeamertemplate{section in toc}[sections numbered]
            \tableofcontents[hideallsubsections]
        \end{frame}
    \section{Perturbaci\'on con MB}
        \begin{frame}
    \frametitle{Consideremeos la siguiente ede}
    hablar del switch en persistencia y extincion
    por amplitud de ruido
\end{frame}
    \section{Propiedades del proceso souluci\'on}
        \begin{frame}
    \frametitle{Existencia de solución única positiva}
    \begin{textblock*}{60mm}(0mm, 25mm)
        \begin{yellowbox}{Teorema}
            \begin{list}{\bullet}{}
                \item
                    $I(0) \in (0, N)$
            \end{list}
            \tcblower
             existe c.s.
            \textbf{solución 
            global única positiva} 
            e 
            \textbf{invariante}
            a EDE(\star)
            $$
                \Pr
                \{
                    I(t) \in (0, N)
                    \ \forall t \geq 0
                \} = 1.
            $$
        \end{yellowbox}
    \end{textblock*}
    \begin{textblock*}{60mm}(65mm, 25mm)
        \begin{align*}
            dI(t) &= 
                I(t)
                \left\{
                    [
                        \beta (N - I(t))
                        -\mu - \gamma
                    ]
                    dt
                \right.
                \\
                 &+
                \left.
                    \sigma (N-I(t))
                    dB_t
                \right\}
                \tag{\star}
        \end{align*}
    \end{textblock*}
\end{frame}
%%%%%%%%%%%%%%%%%%%%%%%%%%%%%%%%%%%%%%%%%%%%%%%%%%%%%%%%%%%%%%%%%%%%%%%%%%%%%%%%
\begin{frame}
    \frametitle{Extinción por ruido ambiental}
        \begin{textblock*}{65mm}(0mm, 25mm)
            \begin{yellowbox}{Teorema}
                \begin{list}{\bullet}{}
                    \item
                        $
                            \displaystyle
                            \sigma ^ 2 
                            >
                            \max
                            \left\{
                                \frac{\beta}{N},
                                \frac{\beta^2}{2(\mu + \gamma)}
                            \right\}
                        $
                \end{list}
                \tcblower
                para toda $I(0)\in (0, N)$ la solución de EDE(\star)
                cumple
                $$
                    \limsup_{t \to \infty} 
                        \frac{1}{t} \log I(t)
                        \leq
                        \underbrace{
                        -\mu - \gamma 
                        + \frac{\beta^2}{2 \sigma ^ 2}
                        }_{<0}
                $$
            \end{yellowbox}
        \end{textblock*}
        \begin{textblock*}{60mm}(65mm, 25mm)
            \begin{align*}
                dI(t) &= 
                    I(t)
                    \left\{
                        [
                            \beta (N - I(t))
                            -\mu - \gamma
                        ]
                        dt
                    \right.
                    \\
                     &+
                    \left.
                        \sigma (N-I(t))
                        dB_t
                    \right\}
                    \tag{\star}
            \end{align*}
        \end{textblock*}
\end{frame}
    \section{%
        Umbral: %
        $   
            R_0 ^ S := 
                R_0 ^ D 
                -%
                f(\sigma, \cdot)
        $
    }%
        \begin{frame}
    \frametitle{Extinción}
        \begin{textblock*}{65mm}(1mm, 20mm)
            \begin{yellowbox}{Teorema (Extinción)}
                \begin{list}{\bullet}{}
                    \item
                        $
                            \displaystyle
                            R_0 ^ S := R_0 ^D
                            - \frac{\sigma ^ 2}{2(\mu + \gamma)} < 1
                        $
                    \item
                        $
                            \displaystyle
                            \sigma^2 \leq \frac{\beta}{N}
                        $
                \end{list}
                \tcblower
                para toda $I(0)\in (0, N)$
                \begin{align*}
                    &\limsup_{t \to \infty} 
                    \frac{1}{t} \log I(t)
                        \leq
                        \kappa,  \quad c.s.
                    \\
                        &\kappa :=
                        \beta N - \mu - \gamma 
                        - \frac{\sigma^2 N^2}{2 \sigma ^ 2}
                        <0 
                \end{align*}
            \end{yellowbox}
        \end{textblock*}
        \begin{textblock*}{62mm}(65mm, 25mm)
            \begin{align*}
                dI(t) &= 
                    I(t)
                    \left\{
                        [
                            \beta (N - I(t))
                            -\mu - \gamma
                        ]
                        dt
                    \right.
                    \\
                     &+
                    \left.
                        \sigma (N-I(t))
                        dB_t
                    \right\}
                    \tag{\star}
            \end{align*}
        \end{textblock*}
\end{frame}
\begin{frame}
    \frametitle{Simulación: Extinción por condición umbral}
    \begin{figure}[h!]
        \centering
        \includegraphics[width=1.01\textwidth, keepaspectratio]%
        {./Figures/IStochastiSIS01.png}
        %\caption{Extinción}
    \end{figure}
\end{frame}
%%%%%%%%%%%%%%%%%%%%%%%%%%%%%%%%%%%%%%%%%%%%%%%%%%%%%%%%%%%%%%%%%%%%%%%%%%%%%%%%

        \input{./Perturbation/persistence.tex}
        \begin{frame}
    \begin{textblock*}{60mm}(1mm, 25mm)
        \frametitle{Distribución estacionaria}
        \begin{greenbox}{Distribución estacionaria}
            \begin{list}{\bullet}{}
                \item
                    $
                        P_{I_0,t}(A)
                        = \Pr[I(t)\in A]
                    $,
                    \\
                    $A \in \mathcal{B}((0, N))$
                \item
                    $
                        \displaystyle
                        \lim_{t \to \infty}
                        P_{I_0,t}(\cdot)
                        = \mathbb{P}_{\infty}(\cdot)
                    $ en distribución.
            \end{list}
        \end{greenbox}
    \end{textblock*}
    \begin{textblock*}{60mm}(61mm, 20mm)
        \only<3>{
            \includegraphics[width=\textwidth, keepaspectratio]%
            {Figures/histo_sigma_1.png}
        }
        \only<4>{
            \includegraphics[width=\textwidth, keepaspectratio]%
            {Figures/histo_sigma_2.png}
        }
    \end{textblock*}
\end{frame}

    \section{Ideas}
        \begin{frame}[plain]
    \frametitle{}
    \only<1>{
        \begin{bibunit}
            \nocite{Jerez2018,Acuna-Zegarra2018}
            \putbib
        \end{bibunit}
    }
    \only<2>{
        \begin{bibunit}
            \nocite{Diaz-Infante2017a}
            \putbib
        \end{bibunit}
    }
\end{frame}
%%%%%%%%%%%%%%%%%%%%%%%%%%%%%%%%%%%%%%%%%%%%%%%%%%%%%%%%%%%%%%%%%%%%%%%%%%%%%%%%
\begin{frame}
    \frametitle{Ideas}
        \begin{textblock*}{50mm}(3mm, 18mm)
            \begin{bluebox}{Modelos}
                \begin{itemize}
                        \item
                            \textcolor<1>{orange}{
                                Procesos
                                reversibles
                                en media
                            }
                        \item
                            \textcolor<2>{orange}{
                                $
                                    \beta_t^H \ H \in (\num{.5}, 1)
                                $
                            }
                        \item
                            \textcolor<3>{orange}{
                                Random Diff. Eq.
                            }
                        \item
                            Random Dynamical Systems
                \end{itemize}
            \end{bluebox}
        \end{textblock*}
%%%%%%%%%%%%%%%%%%%%%%%%%%%%%%%%%%%%%%%%%%%%%%%%%%%%%%%%%%%%%%%%%%%%%%%%%%%%%%%%
        \begin{textblock*}{72mm}(55mm, 18mm)
           \begin{graybox}{
                \only<1>{%
                    $d
                        \varphi_t =
                        (\varphi_e - \varphi_t) dt
                        + \sigma_\varphi dB_t
                    $%
                }%
                \only<2>{
                    M.B. Fraccionario: memoria.
                }
                \only<3>{%
                    parametros son v.a.
                }
            }%
                \only<1>{
                    \begin{bibunit}[apalike]
                        \nocite{Allen2016}
                        \putbib
                    \end{bibunit}
                }
                \only<2>{
                    \begin{bibunit}[apalike]
                        \nocite{Ma2017}
                        \putbib
                    \end{bibunit}
                }
                \only<3>{
                    \begin{bibunit}[apalike]
                        \nocite{Chen-Charpentier2015}
                        \putbib
                    \end{bibunit}
                }
          \end{graybox}
       \end{textblock*}
\end{frame}
%%%%%%%%%%%%%%%%%%%%%%%%%%%%%%%%%%%%%%%%%%%%%%%%%%%%%%%%%%%%%%%%%%%%%%%%%%%%%%%%
\begin{frame}
    \frametitle{Las biblias:}
    \only<1>{
        \begin{bibunit}
            \nocite{Freidlin1998,Khasminskii}
            \putbib
        \end{bibunit}
    }
    \only<2>{
        \begin{bibunit}
            \nocite{Kloeden1995, Kloeden2017}
            \putbib
        \end{bibunit}
    }
\end{frame}
%%%%%%%%%%%%%%%%%%%%%%%%%%%%%%%%%%%%%%%%%%%%%%%%%%%%%%%%%%%%%%%%%%%%%%%%%%%%%%%%%
\begin{frame}
    \frametitle{FIN}
    Código: \texttt{*.tex, *.py}
    \begin{center}
        \href{
            https://github.com/SaulDiazInfante/JXV-thesis-workshop.git%
        }%
        {https://github.com/SaulDiazInfante/JXV-thesis-workshop.git}
    \end{center}
    \only<2>{
        \begin{center}
            \Huge{Gracias!!!}
        \end{center}
        }
\end{frame}
\end{document}

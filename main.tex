\documentclass[compress]{beamer}
\usepackage[spanish, activeacute]{babel}
\usepackage{xcolor}
\usepackage{color}
\usepackage{colortbl}
\usepackage{amsmath}
\usepackage{amssymb}
\usepackage{graphicx}
\usepackage{latexsym}
\usepackage{ucs}
\usepackage[utf8]{inputenc}
\usepackage{wrapfig}
\usepackage{siunitx}
\usepackage{times}
\usepackage{tikz}
\usepackage{verbatim}
\usepackage{multimedia}
\usepackage{hyperref}
\usepackage{thumbpdf}
\usepackage{wasysym}
\usepackage{pgf,pgfarrows,pgfnodes,pgfautomata,pgfheaps,pgfshade}
\usepackage{url}
\usepackage{empheq}
\usepackage{fancybox}
\usepackage{esint}
\usepackage{lipsum}
\usepackage{listings}
\usepackage{mathptmx}
\usepackage{helvet}
\usepackage{tikz}%
\usepackage{circuitikz}
\usepackage{csvsimple}
\usepackage{pgfplots}
\usepackage{multimedia}
\usepackage{media9}
\usepackage{proba}
\usepackage[absolute,overlay]{textpos}
\usepackage{bibunits}
\usepackage{tcolorbox}
\usepackage[
    texcoord,
    grid,
    gridunit=mm,
    gridcolor=red!60,
    subgridcolor=gray!60]{eso-pic}
\usepackage[makeroom]{cancel}
\usepackage{epstopdf}
\usepackage{thumbpdf}
\usepackage{wasysym}
\usepackage{ucs}
\usepackage[utf8]{inputenc}
\usepackage{pgf,pgfarrows,pgfnodes,pgfautomata,pgfheaps,pgfshade}
\usepackage{pgfpages}
\usepackage{verbatim}
\usepackage{fancyvrb}
\usepackage{multimedia}
\usepackage{subcaption}
\usepackage{ulem}
\usepackage{textcomp}
\usepackage{tikz}
\usepackage{listings}
\usepackage{epigraph}
\setlength{\epigraphwidth}{.8\textwidth}
\usepackage{DejaVuSansMono}
%
% Adjust the colours to fit your design
\definecolor{mainthemecolour}{rgb}{0.42,0.48,0.37}
\definecolor{mainthemecolourlight}{rgb}{0.63,0.72,0.57}
\definecolor{mainthemecolourstrong}{rgb}{0.40,0.68,0.18}
\definecolor{mid-gray}{gray}{0.7}
%
\definecolor{greenstrong}{rgb}{0.58,0.77,0.29}
\definecolor{redstrong}{rgb}{0.81,0.22,0.23}
\definecolor{fglisting}{gray}{0.3}
\definecolor{bglisting}{gray}{1}
\definecolor{fgshell}{gray}{1}
\definecolor{bgshell}{gray}{0.1}
\definecolor{bgshelllight}{gray}{0.8}
%
%
\makeatletter
\let\@@magyar@captionfix\relax
\makeatother
%
% Some in-code macros - a bit buggy, but useful
\newcommand{\hl}[1]{\textcolor{greenstrong}{#1}}
\newcommand{\hlErr}[1]{\textcolor{redstrong}{\texttt{#1}}}
\newcommand{\hlOk}[1]{\textcolor{green}{\texttt{#1}}}
\newcommand{\hlInv}[1]{\colorbox{bgshell}{\textcolor{fgshell}{\texttt{#1}}}}
\newcommand{\unhl}[1]{\textcolor{gray}{#1}}
\newcommand{\clda}[0]{$\textcolor{blue}{\lambda}$}
\newcommand{\carr}[0]{$\textcolor{purple}{\rightarrow}$}
\newcommand{\cbind}[0]{\textbf{\texttt{$>\!\!>\!\!=$}}}
\newcommand{\codedots}[0]{\textcolor{mid-gray}{...}}
%%%%%%%%%%%%%%%%%%%%%%%%%%%%%%%%%%%%%%%%%%%%%%%%
\usetheme{elegance}
%\usetheme{INRA}
%\useoutertheme{miniframes}
%-------------------------------------------------------------------------------
% Quelques options pdf
%-------------------------------------------------------------------------------
\hypersetup{
pdfpagemode = , % afficher le pdf en plein écran
pdfauthor   = {Saul Diaz Infante Velasco},%
pdftitle    = {Charla Biomatesis},%
pdfsubject  = {Bio-matematicas},%
pdfkeywords = {Science,Impact},%
pdfcreator  = {PDFLaTeX,emacs,AucTeX, LuaLaTex},%
pdfproducer = {CONACYT-Universidad de Sonora}%
}
\lstnewenvironment{cxxcode}
    {\lstset
        { escapeinside={@}{@}
        , gobble=8
        , showstringspaces=false
        , basicstyle=\color{fglisting}
        , rulecolor=\color{mainthemecolourlight}
        }
    }
    {}

\lstnewenvironment{cxxcodebox}
    {\lstset
        { escapeinside={@}{@}
        , gobble=6
        , showstringspaces=false
        , basicstyle=\color{fglisting}
        , frame=lr
        , rulecolor=\color{mainthemecolourlight}
        }
    }
    {}
\lstnewenvironment{shellcode}
    {\lstset
        { escapeinside={@}{@}
        , gobble=7
        , showstringspaces=false
        , basicstyle=\color{fgshell}
        , backgroundcolor=\color{bgshell}
        }
    }
    {}
%
% Marking points to use in Tikz
\usetikzlibrary{arrows,shapes}
\newcommand{\tikzmark}[1]{\tikz[remember picture] \node[coordinate] (#1) {#1};}
%
% Fragile frames
\newenvironment{xframe}[1][]
  {\begin{frame}[fragile,environment=xframe,#1]}
  {\end{frame}}

\title{Extinci\'on, Persistencia y Comportamiento 
  \\ Umbral en Modelos Compartimentales}
\subtitle{Estoc\'asticmente Perturbados}
\author{Sa\'ul D\'iaz Infante Velasco}
\institute{\color{white}
    CONACYT-Universidad de Sonora \\
    sauldiazinfante@gmail.com \qquad
    https://sauldiazinfantevelasco.wordpress.com%
} %
\date{
  \footnotesize\color{mainthemecolour} UNAM, 
  Juriquilla, Queretaro  
  \\
  \today. }
%%%%%%%%%%%%%%%%%%%%%%%%%%%%%%%%%%%%%%%%%%%%%%%%%%%%%%%%%%%%%%%%%%%%%%%%%%%%%%%%
\makeatletter
\let\@@magyar@captionfix\relax
\makeatother
\input{setup}
\defaultbibliography{main}
\defaultbibliographystyle{abbrv}
%%%%%%%%%%%%%%%%%%%%%%%%%%%%%%%%%%%%%%%%%%%%%%%%%%%%%%%%%%%%%%%%%%%%%%%%%%%%%%%%
\begin{document}
    \maketitle
    \section{Introducci\'on}
        \begin{frame}
    \frametitle{Para fijar ideas}
    \begin{textblock*}{70mm}(5mm, 20mm)
        \begin{equation*}
            \begin{aligned}
                \dot{S}(t) &=
                    \Lambda - \mu S(t)
                    - \textcolor<4->{orange}{\beta}
                        S(t) I(t)
                    - \delta S(t)
                    \\
                \dot{I}(t) &=
                    \textcolor<4->{orange}{\beta}
                        S(t) I(t)
                    -(\mu + \gamma + \varepsilon) I(t)
                    \\
                \only<1-2>{
                    \dot{R}(t) &=
                        \gamma I(t)
                        - \mu R(t)
                        + \delta S(t)
                }
            \end{aligned}
        \end{equation*}
    \end{textblock*}
%%
    \only<2-7>{
        \begin{textblock*}{50mm}(5mm, 60mm)
            \begin{tcolorbox}[%
                space to upper,
                skin=bicolor,
                colbacklower=black!75,
                collower=white,
                title={Umbral determinista},
                halign=center,
                valign=center,
                bottom=2mm,
                height=35mm
            ]
                \begin{align*}
                    \mathcal{R}_0 &=
                        \frac{\beta \Lambda}{%
                            (\mu + \gamma + \varepsilon )%
                            (\mu + \delta)%
                        }%
                    \\
                    \mathcal{R}_0  & < 1
                        \ \Rightarrow
                            \ FDE: \text{ (g.a.s)}
                    \\
                    \mathcal{R}_0  & > 1
                        \ \Rightarrow
                            \ EE: \text{\quad (g.a.s)}
                \end{align*}
            \end{tcolorbox}
        \end{textblock*}
    }
    \only<5->{
        \begin{textblock*}{42mm}(80mm, 25mm)
             \begin{tcolorbox}
                 $
                     \beta d t \rightsquigarrow
                     \beta d t + \sigma dB_t
                 $
             \end{tcolorbox}
         \end{textblock*}
    }
%     %
    \only<6->{
        \begin{textblock*}{50mm}(40mm, 40mm)
            \begin{align*}
                \dot{S}(t) &=
                    \Lambda - \mu S(t)
                    - \beta S(t) I(t)
                    - \delta S(t)
                    -
                    \hl{%
                      \sigma S(t) I(t)
                       dB_t}
                    \\
                \dot{I}(t) &=
                    \beta S(t) I(t)
                    - (\mu + \gamma + \varepsilon) I(t)
                    + \hl{\sigma S(t) I(t) dB_t}
            \end{align*}
        \end{textblock*}
    }
    \only<7>{
        \begin{textblock*}{50mm}(70mm, 60mm)
            \begin{tcolorbox}[%
                space to upper,
                skin=bicolor,
                colbacklower=black!75,
                collower=white,
                title={Umbral estoc\'astico},
                halign=center,
                valign=center,
                bottom=2mm,
                height=35mm
            ]
                \begin{align*}
                    \mathcal{R}_0 ^ S &= \text{ ?}
                    \\
                    \mathcal{R}_0 ^ S &< 1
                    \ \Rightarrow \text{ extinci\'on}
                    \\
                    \mathcal{R}_0 ^ S &> 1
                    \ \Rightarrow \text{ persistencia}
                \end{align*}
            \end{tcolorbox}
        \end{textblock*}
    }
    \only<8>{
        \begin{textblock*}{105mm}(20mm, 57mm)
            \begin{tcolorbox}[title=Ver:]
                \begin{bibunit}[apalike]
                    \nocite{Zhang2018a}
                    \putbib
                \end{bibunit}
            \end{tcolorbox}
        \end{textblock*}
    }
\end{frame}

        \begin{frame}
    \frametitle{?`Cuando considerar Modelos Estoc\'asticos?}
    \begin{textblock*}{45mm}(3mm, 25mm)
        \begin{greenbox}{Sean importantes}
            \begin{itemize}
                \item
                    Poblaciones peque\~nas
                \item
                    \textcolor<2>{orange}{
                        Variabilidad demogr\'afica
                    }
                \item
                    \textcolor<3>{orange}{
                        Variabilidad ambiental
                    }
            \end{itemize}
        \end{greenbox}
    \end{textblock*}
    \begin{textblock*}{60mm}(60mm, 25mm)
        \begin{bluebox}{
            \only<1>{
                Seg\'un:
            }
            \only<2->{
                Ejemplo
            }%
        }
        \only<1>{
            \begin{bibunit}[apalike]
                \nocite{Allen2017}
                \putbib
            \end{bibunit}
        }
        \only<2->{
            Transmisi\'on, recuperaci\'on,
            nacimientos, muertes.
        }
        \only<3->{
            \tcblower
            Condiciones territoriales,
            acu\'aticas: enfermedades
            vectoriales, zoonóticas
            transmitidas por alimentos.
        }
        \end{bluebox}
    \end{textblock*}
\end{frame}

        \input{./introduction/alternativas.tex}
        \begin{frame}
    \frametitle{Objetivo}
    \begin{textblock*}{80mm}(25mm, 40mm)
            \begin{yellowbox}{%
                Ilustrar las ideas de 
                $
                    \varphi dt 
                    \rightsquigarrow 
                    \varphi dt 
                    + 
                    \sigma dB_t
                $
            }
                \begin{list}{$\bullet$}{}
                    \item
                        Modelación
                    \item
                        Análisis y Simulación 
                    \item
                        Ideas al aire
                \end{list}
            \end{yellowbox}
    \end{textblock*}
\end{frame}
        \begin{frame}{Esquema de Charla}
            \setbeamertemplate{section in toc}[sections numbered]
            \tableofcontents[hideallsubsections]
        \end{frame}
    \section{Perturbaci\'on con MB}
        \begin{frame}
    \frametitle{Consideremeos la siguiente ede}
    hablar del switch en persistencia y extincion
    por amplitud de ruido
\end{frame}
    \section{Propiedades del proceso souluci\'on}
        %\subsection{$\exists$ Positividad e Invariancia}
        %\subsection{Extinci\'on}
        %\subsection{Persistencia}
    \section{
        Umbral: %
        $
            \mathcal{R}_0^S:= 
                \mathcal{R}_0^D 
                    - f(\text{noise})
                    
        $
    }
\end{document}

\documentclass{beamer}
\usepackage[spanish, activeacute]{babel}
\usepackage{xcolor}
\usepackage{color}
\usepackage{colortbl}
\usepackage{amsmath}
\usepackage{amssymb}
\usepackage{graphicx}
\usepackage{latexsym}
\usepackage{ucs}
\usepackage[utf8]{inputenc}
\usepackage{siunitx}
\usepackage{times}
\usepackage{tikz}
\usepackage{verbatim}
\usepackage{pgf,pgfarrows,pgfnodes,pgfautomata,pgfheaps,pgfshade}
\usepackage{listings}
\usepackage{esint}
\usepackage{lipsum}
\usepackage{mathptmx}
\usepackage{helvet}
\usepackage{tikz}%
\usepackage{csvsimple}
\usepackage{pgfplots}
\usepackage{proba}
\usepackage[absolute, overlay]{textpos}
\usepackage{bibunits}
\usepackage{tcolorbox}
\usepackage[
    texcoord,
    grid,
    gridunit=mm,
    gridcolor=red!60,
    subgridcolor=gray!60]{eso-pic}
\usepackage{epigraph}
\setlength{\epigraphwidth}{.8\textwidth}
\usepackage{DejaVuSansMono}
%
\definecolor{mainthemecolour}{rgb}{0.42,0.48,0.37}
\definecolor{mainthemecolourlight}{rgb}{0.63,0.72,0.57}
\definecolor{mainthemecolourstrong}{rgb}{0.40,0.68,0.18}
\definecolor{mid-gray}{gray}{0.7}
%
\definecolor{greenstrong}{rgb}{0.58,0.77,0.29}
\definecolor{redstrong}{rgb}{0.81,0.22,0.23}
\definecolor{fglisting}{gray}{0.3}
\definecolor{bglisting}{gray}{1}
\definecolor{fgshell}{gray}{1}
\definecolor{bgshell}{gray}{0.1}
\definecolor{bgshelllight}{gray}{0.8}
%
%
\makeatletter
\let\@@magyar@captionfix\relax
\makeatother
%
% Some in-code macros - a bit buggy, but useful
\newcommand{\hl}[1]{\textcolor{greenstrong}{#1}}
\newcommand{\hlErr}[1]{\textcolor{redstrong}{\texttt{#1}}}
\newcommand{\hlOk}[1]{\textcolor{green}{\texttt{#1}}}
\newcommand{\hlInv}[1]{\colorbox{bgshell}{\textcolor{fgshell}{\texttt{#1}}}}
\newcommand{\unhl}[1]{\textcolor{gray}{#1}}
\newcommand{\clda}[0]{$\textcolor{blue}{\lambda}$}
\newcommand{\carr}[0]{$\textcolor{purple}{\rightarrow}$}
\newcommand{\cbind}[0]{\textbf{\texttt{$>\!\!>\!\!=$}}}
\newcommand{\codedots}[0]{\textcolor{mid-gray}{...}}
%%%%%%%%%%%%%%%%%%%%%%%%%%%%%%%%%%%%%%%%%%%%%%%%
\usetheme{elegance}
%%%%%%%%%%%%%%%%%%%%%%%%%%%%%%%%%%%%%%%%%%%%%%%%%%%%%%%%%%%%%%%%%%%%%%%%%%%%%%%%
\tcbuselibrary{skins, breakable}
%
\newtcolorbox{greenbox}[1]{%
        colback = green!5!white,
        colframe = green!55!black,
        fonttitle = \bfseries,
        title = #1
    }
\newtcolorbox{bluebox}[1]{%
        colback = blue!5!white,
        colframe = blue!55!black,
        fonttitle = \bfseries,
        title = #1
    }
%
\newtcolorbox{graybox}[1]{%
        colback = gray!5!white,
        colframe = gray!55!black,
        fonttitle = \bfseries,
        title = #1
    }
%
\newtcolorbox{yellowbox}[1]{%
        colback = yellow!5!white,
        colframe = yellow!55!black,
        fonttitle = \bfseries,
        title = #1
    }
%
\hypersetup{
    pdfpagemode = , % afficher le pdf en plein écran
    pdfauthor   = {Saul Diaz Infante Velasco},%
    pdftitle    = {Charla Biomatesis},%
    pdfsubject  = {Bio-matematicas},%
    pdfkeywords = {Science,Impact},%
    pdfcreator  = {PDFLaTeX,emacs,AucTeX, LuaLaTex},%
    pdfproducer = {CONACYT-Universidad de Sonora}%
}
%
\usetikzlibrary{arrows,shapes}
\newcommand{\tikzmark}[1]{\tikz[remember picture] \node[coordinate] (#1) {#1};}
%
% Fragile frames
\newenvironment{xframe}[1][]
  {\begin{frame}[fragile,environment=xframe,#1]}
  {\end{frame}}

\title{%
    Extinci\'on, Persistencia y Comportamiento % 
    \\ Umbral en Modelos Compartimentales %
}%
\subtitle{Estoc\'asticmente Perturbados}
\author{Sa\'ul D\'iaz Infante Velasco}
\institute{
    \color{white}
    CONACYT-Universidad de Sonora \\
    sauldiazinfante@gmail.com \qquad
    https://sauldiazinfantevelasco.wordpress.com%
} %
\date{
    \footnotesize
    \color{mainthemecolour} 
    UNAM, 
    Juriquilla, Queretaro  
    \\
    \today. }
%%%%%%%%%%%%%%%%%%%%%%%%%%%%%%%%%%%%%%%%%%%%%%%%%%%%%%%%%%%%%%%%%%%%%%%%%%%%%%%%
\makeatletter
\let\@@magyar@captionfix\relax
\makeatother
%%%%%%%%%%%%%%%%%%%%%%%%%%%%%%%%%%%%%%%%%%%%%%%%%%%%%%%%%%%%%%%%%%%%%%%%%%%%%%%%
\def\Q#1#2{\frac{\partial #1}{\partial #2}}
\usetikzlibrary{arrows,shapes}
\pgfplotsset{compat=1.14}
%%%%%%%%%%%%%%%%%%%%%%%%%%%%%%%%%%%%%%%%%%%%%%%%%%%%%%%%%%%%%%%%%%%%%%%%%%%%%%%%
%------------------------------------Theorems 
\theoremstyle{plain} % default
\newtheorem{Teorema}{Teorema}
\newtheorem{Ejemplo}{Ejemplo}
\theoremstyle{definition}
\newtheorem{Definicion}{Definici\'on}
\newtheorem{Corolario}{Corolario}
\newtheorem{Proposicion}{Proposici\'on}
\newtheorem{Prueba}{Prueba}
\theoremstyle{definition}
\newtheorem{definicion}{Definici\'on}
\newtheorem{lema}{Lema}
%-----------------------------ExtrasDeTercerPresentacion
%--------------------------------Fancyboxes-------------------------------------
\definecolor{myblue}{rgb}{.8, .8, 1}
\definecolor{shadecolor}{cmyk}{0,0,0.41,0}
\newcommand*\mybluebox[1]{%
    \colorbox{myblue}{\hspace{1em}#1\hspace{1em}}
}
\newcommand*\myyellowbox[1]{%
    \colorbox{darkyellow}{\hspace{1em}#1\hspace{1em}}
}
%--------------------------------------------------------------------------
\definecolor{shadecolor}{cmyk}{0,0,0.41,0}
\definecolor{light-blue}{cmyk}{0.25,0,0,0}
\newsavebox{\mysaveboxM} % M for math
\newsavebox{\mysaveboxT} % T for text
\newcommand*\Garybox[2][Example]{%
    \sbox{\mysaveboxM}{#2}%
        \sbox{\mysaveboxT}{\fcolorbox{black}{light-blue}{#1}}%
            \sbox{\mysaveboxM}{%
    \parbox[b][\ht\mysaveboxM+.5\ht\mysaveboxT+.5\dp\mysaveboxT][b]{%
        \wd\mysaveboxM}{#2}%
    }%
    \sbox{\mysaveboxM}{%
        \fcolorbox{black}{shadecolor}{%
        \makebox[\linewidth-10em]{\usebox{\mysaveboxM}}%
        }%
    }%
    \usebox{\mysaveboxM}%
    \makebox[0pt][r]{%
        \makebox[\wd\mysaveboxM][c]{%
            \raisebox{\ht\mysaveboxM-0.5\ht\mysaveboxT
            +0.5\dp\mysaveboxT-0.5\fboxrule}{\usebox{\mysaveboxT}}%
        }%
    }%
}
\newcommand\Fontvi{\fontsize{7}{7.2}\selectfont}
%%%%%%%%%%%%%%%%%%%%%%%%%%%%%%%%%%%%%%%%%%%%
\definecolor{kugreen}{RGB}{50,93,61}
\definecolor{kugreenlys}{RGB}{132,158,139}
\definecolor{kugreenlyslys}{RGB}{173,190,177}
\definecolor{kugreenlyslyslys}{RGB}{214,223,216}
\definecolor{greenArea}{RGB}{124,252,124}
\definecolor{hellmagenta}{rgb}{1,0.75,0.9}
\definecolor{hellcyan}{rgb}{0.75,1,0.9}
\definecolor{hellgelb}{rgb}{1,1,0.8}
\definecolor{colKeys}{rgb}{0,0,1}
\definecolor{colIdentifier}{rgb}{0,0,0}
\definecolor{colComments}{rgb}{1,0,0}
\definecolor{colString}{rgb}{0,0.5,0}
\definecolor{darkyellow}{rgb}{1,0.9,0}
\setbeamercovered{transparent}
\lstset{%
    language=[AlLaTeX]TEX,%
    float=hbp,%
    basicstyle=\ttfamily\small, %\usepackage{cir}
    identifierstyle=\color{colIdentifier}, %
    keywordstyle=\color{colKeys}, %
    stringstyle=\color{colString}, %
    commentstyle=\color{colComments}, %
    columns=flexible, %
    tabsize=3, %
    frame=single, %
    extendedchars=true, %
    showspaces=false, %
    showstringspaces=false, %
    numbers=left, %
    numberstyle=\tiny, %
    breaklines=true, %
    backgroundcolor=\color{hellgelb}, %
    breakautoindent=true, %
    captionpos=b,%
    xleftmargin=18pt,%
    xrightmargin=\fboxsep%
}
\pgfplotsset{
    left segments/.code={\pgfmathsetmacro\leftsegments{#1}},
    left segments=3,
    left/.style args={#1:#2}{
        ybar interval,
        domain=#1:#2,
        samples=\leftsegments+1,
        x filter/.code=\pgfmathparse{\pgfmathresult}
       }
}
\DeclareMathOperator{\sign}{sgn}
\newcommand{\innerprod}[2]{\left\langle#1, #2\right\rangle}
\newcommand\bound{10} % bound number of points on each side of N
\newcommand\labelnum[3][]{
    \begin{scope}[font=\footnotesize,x=.3cm,#1]
      \foreach \mypt in {0,#2,...,\bound}{
        \draw(\mypt,0)circle[radius=2pt];
        \draw(-\mypt,0)circle[radius=2pt];
      }
      \draw(-\bound-5,0)--(\bound+5,0) node[pos=0, left]{$t$};
      \node(start)[at={(-\bound-4,0)},label=below:{$t_0=0$}]{$[$};
      \node(end)[at={(\bound+4,0)},label=below:{$T=Nh$}]{$]$};
      \node[%
          at={($(start)!.319!(end)$)},
          label=below:{
               $\underbrace{}_{h}$
            }%
            ]{\vphantom{$[$}};
      \node[at={($(start)!.57!(end)$)},label=below:{$t_{n+1}$}]{\vphantom{$[$}};
      \filldraw(0,0)circle[radius=2pt];
      \node[at={(-\bound-2,0)},above]{$\cdots$};
      \node[at={(\bound+2,0)},above]{$\cdots$};
      \node[at={(0,0)},above=5pt]{#3};
    \end{scope}
}
\usepackage{remreset}
\makeatletter
\@removefromreset{subsection}{section}
\makeatother
\setcounter{subsection}{1}
\defaultbibliography{main}
\defaultbibliographystyle{abbrv}
%%%%%%%%%%%%%%%%%%%%%%%%%%%%%%%%%%%%%%%%%%%%%%%%%%%%%%%%%%%%%%%%%%%%%%%%%%%%%%%%
\begin{document}
    \maketitle
    \section{Introducci\'on}
        \begin{frame}
    \frametitle{Para fijar ideas}
    \begin{textblock*}{70mm}(5mm, 20mm)
        \begin{equation*}
            \begin{aligned}
                \dot{S}(t) &=
                    \Lambda - \mu S(t)
                    - \textcolor<4->{orange}{\beta}
                        S(t) I(t)
                    - \delta S(t)
                    \\
                \dot{I}(t) &=
                    \textcolor<4->{orange}{\beta}
                        S(t) I(t)
                    -(\mu + \gamma + \varepsilon) I(t)
                    \\
                \only<1-2>{
                    \dot{R}(t) &=
                        \gamma I(t)
                        - \mu R(t)
                        + \delta S(t)
                }
            \end{aligned}
        \end{equation*}
    \end{textblock*}
%%
    \only<2-7>{
        \begin{textblock*}{50mm}(5mm, 60mm)
            \begin{tcolorbox}[%
                space to upper,
                skin=bicolor,
                colbacklower=black!75,
                collower=white,
                title={Umbral determinista},
                halign=center,
                valign=center,
                bottom=2mm,
                height=35mm
            ]
                \begin{align*}
                    \mathcal{R}_0 &=
                        \frac{\beta \Lambda}{%
                            (\mu + \gamma + \varepsilon )%
                            (\mu + \delta)%
                        }%
                    \\
                    \mathcal{R}_0  & < 1
                        \ \Rightarrow
                            \ FDE: \text{ (g.a.s)}
                    \\
                    \mathcal{R}_0  & > 1
                        \ \Rightarrow
                            \ EE: \text{\quad (g.a.s)}
                \end{align*}
            \end{tcolorbox}
        \end{textblock*}
    }
    \only<5->{
        \begin{textblock*}{42mm}(80mm, 25mm)
             \begin{tcolorbox}
                 $
                     \beta d t \rightsquigarrow
                     \beta d t + \sigma dB_t
                 $
             \end{tcolorbox}
         \end{textblock*}
    }
%     %
    \only<6->{
        \begin{textblock*}{50mm}(40mm, 40mm)
            \begin{align*}
                \dot{S}(t) &=
                    \Lambda - \mu S(t)
                    - \beta S(t) I(t)
                    - \delta S(t)
                    -
                    \hl{%
                      \sigma S(t) I(t)
                       dB_t}
                    \\
                \dot{I}(t) &=
                    \beta S(t) I(t)
                    - (\mu + \gamma + \varepsilon) I(t)
                    + \hl{\sigma S(t) I(t) dB_t}
            \end{align*}
        \end{textblock*}
    }
    \only<7>{
        \begin{textblock*}{50mm}(70mm, 60mm)
            \begin{tcolorbox}[%
                space to upper,
                skin=bicolor,
                colbacklower=black!75,
                collower=white,
                title={Umbral estoc\'astico},
                halign=center,
                valign=center,
                bottom=2mm,
                height=35mm
            ]
                \begin{align*}
                    \mathcal{R}_0 ^ S &= \text{ ?}
                    \\
                    \mathcal{R}_0 ^ S &< 1
                    \ \Rightarrow \text{ extinci\'on}
                    \\
                    \mathcal{R}_0 ^ S &> 1
                    \ \Rightarrow \text{ persistencia}
                \end{align*}
            \end{tcolorbox}
        \end{textblock*}
    }
    \only<8>{
        \begin{textblock*}{105mm}(20mm, 57mm)
            \begin{tcolorbox}[title=Ver:]
                \begin{bibunit}[apalike]
                    \nocite{Zhang2018a}
                    \putbib
                \end{bibunit}
            \end{tcolorbox}
        \end{textblock*}
    }
\end{frame}

        \begin{frame}
    \frametitle{?`Cuando considerar Modelos Estoc\'asticos?}
    \begin{textblock*}{45mm}(3mm, 25mm)
        \begin{greenbox}{Sean importantes}
            \begin{itemize}
                \item
                    Poblaciones peque\~nas
                \item
                    \textcolor<2>{orange}{
                        Variabilidad demogr\'afica
                    }
                \item
                    \textcolor<3>{orange}{
                        Variabilidad ambiental
                    }
            \end{itemize}
        \end{greenbox}
    \end{textblock*}
    \begin{textblock*}{60mm}(60mm, 25mm)
        \begin{bluebox}{
            \only<1>{
                Seg\'un:
            }
            \only<2->{
                Ejemplo
            }%
        }
        \only<1>{
            \begin{bibunit}[apalike]
                \nocite{Allen2017}
                \putbib
            \end{bibunit}
        }
        \only<2->{
            Transmisi\'on, recuperaci\'on,
            nacimientos, muertes.
        }
        \only<3->{
            \tcblower
            Condiciones territoriales,
            acu\'aticas: enfermedades
            vectoriales, zoonóticas
            transmitidas por alimentos.
        }
        \end{bluebox}
    \end{textblock*}
\end{frame}

        \begin{frame}
    \frametitle{Alternativas}
%%%%%%%%%%%%%%%%%%%%%%%%%%%%%%%%%%%%%%%%%%%%%%%%%%%%%%%%%%%%%%%%%%%%%%%%%%%%%%%%
    \begin{textblock*}{50mm}(3mm, 18mm)
        \begin{bluebox}{Modelos}
            \begin{itemize}
                \item
                    \textcolor<1>{orange}{(D/C)--TMCs}
                \item
                    \textcolor<2,3,4,5,7>{orange}{
                    \textbf<4-5,7>{%
                        Perturbación de parámetros
                    }
                }
                \begin{itemize}
                    \item
                        \textcolor<4>{orange}{
                            Procesos
                            reversibles
                            en media
                        }
                    \item
                        \textcolor<5>{orange}{
                            $
                                \beta_t^H \ H \in (\num{.5}, 1)
                            $
                        }
                \end{itemize}
                \item
                    \textcolor<6>{orange}{
                        Random Diff. Eq.
                    }
            \end{itemize}
        \end{bluebox}
    \end{textblock*}
%%%%%%%%%%%%%%%%%%%%%%%%%%%%%%%%%%%%%%%%%%%%%%%%%%%%%%%%%%%%%%%%%%%%%%%%%%%%%%%%
   \begin{textblock*}{72mm}(55mm, 18mm)
       \begin{graybox}
            {%
                \only<1>{%
                    $\mathbf{MC}
                    +\mathbf{ME}$
                    $\to$ $SDE$%
                }%
                \only<2>{%
                    $\varphi dt \rightsquigarrow \varphi dt+ \sigma dB_t$%
                }%
                \only<3>{%
                    $\varphi dt \rightsquigarrow \varphi dt+F(x)dB_t$%
                }%
                \only<4>{%
                    $d
                        \varphi_t =
                        (\varphi_e - \varphi_t) dt
                        + \sigma_\varphi dB_t
                    $%
                }%
                \only<5>{
                    M.B. Fraccionario: memoria.
                }
                \only<6>{%
                    parametros son v.a.
                }%
                \only<7>{%
                    $\varphi dt \rightsquigarrow \varphi dt+ \sigma dB_t$%
                }%
            }%Block titles
            \only<1>{
                \begin{bibunit}[abbrv]
                    \nocite{Allen2017}
                    \putbib
                \end{bibunit}
            }
            \only<2,7>{
                \begin{bibunit}[apalike]
                    \nocite{Gray2011}
                    \putbib
                \end{bibunit}
            }
            \only<3>{
                \begin{bibunit}[apalike]
                    \nocite{Schurz2015}
                    \putbib
                \end{bibunit}
            }
            \only<4>{
                \begin{bibunit}[apalike]
                    \nocite{Allen2016}
                    \putbib
                \end{bibunit}
            }
            \only<5>{
                \begin{bibunit}[apalike]
                    \nocite{Ma2017}
                    \putbib
                \end{bibunit}
            }
            \only<6>{
                \begin{bibunit}[apalike]
                    \nocite{Chen-Charpentier2015}
                    \putbib
                \end{bibunit}
            }
        \end{graybox}
   \end{textblock*}
   \begin{textblock*}{50mm}(3mm, 68mm)
        \begin{greenbox}{Herramientas}
            \begin{list}{\bullet}{}
                \item
                    Gillespie
                \item
                    \small{Kloeden}-\normalsize{Methods}
                \item
                    Hermite-PC
            \end{list}
        \end{greenbox}
    \end{textblock*}
\end{frame}

        \begin{frame}
    \frametitle{Objetivo}
    \begin{textblock*}{80mm}(25mm, 40mm)
            \begin{yellowbox}{%
                Ilustrar las ideas de 
                $
                    \varphi dt 
                    \rightsquigarrow 
                    \varphi dt 
                    + 
                    \sigma dB_t
                $
            }
                \begin{list}{$\bullet$}{}
                    \item
                        Modelación
                    \item
                        Análisis y Simulación 
                    \item
                        Ideas al aire
                \end{list}
            \end{yellowbox}
    \end{textblock*}
\end{frame}
        %
        \begin{frame}{Esquema de Charla}
            \setbeamertemplate{section in toc}[sections numbered]
            \tableofcontents[hideallsubsections]
        \end{frame}
    \section{Perturbaci\'on con MB}
        \begin{frame}
    \frametitle{Modelo de jugete}
    \begin{textblock*}{60mm}(2mm, 25mm)
        \only<1>{
            \begin{align*}
                \dfrac{dS(t)}{dt} &= 
                    \mu N - \beta S(t) I(t) + \gamma I(t) - \mu S(t),
                \\
                \dfrac{I(t)}{dt} &=
                    \beta S(t) I(t) - (\mu + \gamma) I(t),
            \end{align*}
        }
        \only<2-7>{
            \begin{align*}
                d I(t) &= 
                    \left[ 
                        \beta S(t) I(t) - (\mu + \gamma) I(t)
                    \right] dt
            \end{align*}
        }
        \only<7-9>{
            \begin{align*}
                \dfrac{dS(t)}{dt} &= 
                    [\mu N - \beta S(t) I(t) + \gamma I(t) - \mu S(t)] dt
                    - \sigma S(t)I(t)dB_t
                \\
                \dfrac{I(t)}{dt} &=
                    [\beta S(t) I(t) - (\mu + \gamma) I(t)]dt,
                    + \sigma S(t)I(t)dB_t
            \end{align*}
        }
        \only<9>{
            \begin{align*}
                \dfrac{dI(t)}{dt} &= 
                    I(t)
                    \left(
                        [
                            \beta (N - I(t))
                            -\mu - \gamma
                        ]
                    \right)
                     dt,
                    -\sigma (N-I(t))dB_t
            \end{align*}
        }
    \end{textblock*}
    \only<8>{
        \begin{textblock*}{42mm}(20mm, 37mm)
             \begin{tcolorbox}
                 $
                     \beta d t \rightsquigarrow
                     \beta d t + \sigma dB_t
                 $
             \end{tcolorbox}
         \end{textblock*}
    }
%%%%%%%%%%%%%%%%%%%%%%%%%%%%%%%%%%%%%%%%%%%%%%%%%%%%%%%%%%%%%%%%%%%%%%%%%%%%%%%%
    \begin{textblock*}{60mm}(70mm, 25mm)
        \only<2-6>{
            \begin{itemize}
                \item<3-6>
                    $
                        (
                            \Omega,
                            \mathcal{F},
                            \{\mathcal{F}\}_{t \geq 0},
                            \mathbb{P}
                        )
                   $,
                \item<4-6>
                    $B_t$ M.B.
                \item<5-6>
                    $\beta S(t)I(t) dt$ nuevas infecciones en $[t,t+dt)$
                \item<6>
                    $\beta dt$, contactos potencialmente infecciosos
            \end{itemize}
        }
    \end{textblock*}
\end{frame}
    \section{Propiedades del proceso souluci\'on}
        %\subsection{$\exists$ Positividad e Invariancia}
        %\subsection{Extinci\'on}
        %\subsection{Persistencia}
    \section{%
        Par\'ametro Umbral: %
        $
            R_0 ^ S := 
                R_0 ^ D 
                -%
                f(\text{noise})
        $
    }%
\end{document}
